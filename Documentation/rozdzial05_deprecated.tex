\chapter{Wdrożenie aplikacji}
\section{Wdrożenie części serwerowej}
\paragraph{RestAPI}
% Для развертывая серверной части вместе с базами данных MongoDB и Redis достаточно двух комманд.
% Должны быть установлены docker и docker-compose.

% Первая команда создает 3 образа: образ базы даннх MongoDB, образ базы данных Redis и образ серверной части приложения написаной на Go:
Do wdrożenia części serwerowej wraz z bazami danych MongoDB i Redis wystarczy wykonać dwa polecenia.
Musi być zainstalowany docker i Docker-compose.

Pierwsze polecenie tworzy 3 obrazy: obraz bazy danych dunnh MongoDB, Obraz bazy danych Redis i obraz zaplecza aplikacji napisany w Go:

\texttt{\$ docker-compose build}
% Вторая команда запускает созданные ранее образы в оперделенной последовательности, а также создает каналы для взаимодействия приложений из контейнеров.
Drugie polecenie uruchamia wcześniej utworzone obrazy w określonej kolejności, a także tworzy kanały do interakcji części aplikacji między kontenerami oraz środowyskiem wewnętrznym.

\texttt{\$ docker-compose up}

% Для управления контейнерами ипользуется файл docker-compose:
Do zarządzania kontenerami służy plik \texttt{docker-compose.yml} \ref{list:docker-compose.yml}:
\begin{lstlisting}[label=list:docker-compose.yml,caption=docker-compose.yml,basicstyle=\tiny\ttfamily]
    version: "3.5" # Use version 3.5 syntax
    services: # Here we define our service(s)
        db:
            container_name: mongoDB-elcharge # Container name
            image: mongo # image name to start/build
            ports: # Port mapping
                - "2717:27017"
            volumes: # Volume binding
                - "~/example:/data/db"
        cache-db:
            container_name: redis-elcharge # Container name
            image: redis
            ports:
                - "63799:6379"
            volumes: # Volume binding
                - "/opt/redis/data:/data"
        golang-restapi: # The name of the service
            build:
                context: .
                dockerfile: Dockerfile # Location of our Dockerfile
            image: despenrado/golang-restapi-elcharge:prod.0.1
            container_name: restapi-elcharge # Container name
            depends_on: # start after
                - cache-db
                - db
            ports:
                - "8081:8081"
            links: # list mapping: service_name:name_how_will_see_your_program
                - "db:mymongo"
                - "cache-db:myredis"
    
\end{lstlisting}

% Для этапов устновки зависимостей, компиляции и самого создания образа серверной части приложения использется Dockerfile:
Plik Dockerfile \ref{list:dockerfile} jest używany do etapów instalacji zależności, kompilacji i samego tworzenia obrazu zaplecza aplikacji:
\begin{lstlisting}[label=list:dockerfile,caption=Dockerfile,basicstyle=\tiny\ttfamily]
    FROM golang:1.15 AS builder
    WORKDIR /go/src/github.com/Ddespenrado/ElCharge/RestAPI/
    COPY . .
    RUN go mod tidy
    RUN CGO_ENABLED=0 GOOS=linux GOARCH=amd64 go build -a -tags netgo -ldflags '-w' -o restapi .
    
    FROM scratch
    EXPOSE 8081
    WORKDIR /root/
    COPY deploy/apiserver.yaml .
    COPY --from=builder /go/src/github.com/Ddespenrado/ElCharge/RestAPI/restapi .
    ENTRYPOINT ["./restapi","-config=apiserver.yaml"]
\end{lstlisting}

Plikiem konfiguracyjnym aplikacji, w przypadku używania dockera jest plik \texttt{RestAPI/deploy/apiserver.yaml}.


\section{Instalacja i uruchomienie aplikacji mobilnej}
Najpierw trzeba skopiować plik \texttt{elCharge.apk} do urządzenia z systemem Android, znaleźć ten plik, otworzyć (kliknąć na niego) oraz zgodzić się na instalację.
Uruchomić aplikację \texttt{TestApp}.