\chapter*{Skróty i definicje}\mbox{}\pdfbookmark[0]{Skróty}{skroty.1}
\label{sec:skroty}
\noindent
\begin{description}[labelwidth=*]
  \item [API] (ang. \ \emph{Application Programming Interface}) -- interfejs programowania aplikcji.
  \item [JSON] (ang. \ \emph{JavaScript Object Notation}) -- tekstowy format do wymiany danych, którego podstawą jest Java Scrypt.
  \item [BSON] (ang. \ \emph{Binary JavaScript Object Notation}) -- binarny zapis formatu JSON.
  \item [JWT] (ang. \ \emph{JSON Web Token}) -- standard tworzenia tokenów dostępu.
  \item [SQL] (ang. \ \emph{Structured Query Language}) -- język programowania strukturalnych zapytań. Najczęściej używa się go do skutecznego zapisywania danych, wyszukiwania, zmiany, pobierania oraz usuwania danych z bazy. 
  \item [NoSQL] (ang. \ \emph{Not only SQL}) -- szereg podejść, których celem jest tworzenie systemów zarządzania bazami danych, które mają dużą różnicę w porównaniu do modeli tradycyjnych relacyjnych baz danych.
  \item [SDK] (ang. \ \emph{software development kit}) -- narzędzia programistyczne, umożliwiające tworzenie aplikacji w ramach określonego pakietu oprogramowania.
  \item [Rest] (ang. \ \emph{Representational state transfer}) -- styl architektury oprogramowania przeznaczony do systemów rozproszonych. Zwykle używany do budowy usług internetowych.
  \item [URL] (ang. \ \emph{Uniform Resource Locator}) -- standardem zapisu linków do obiektów w Internecie.
  \item [YAML] (ang. \ \emph{Yet Another Markup Language}) -- język znaczników.
  \item [UUID] (ang. \ \emph{universally unique identifier}) -- standard identyfikacji stosowany w tworzeniu oprogramowania.
  \item [HTTP] (ang. \ \emph{HyperText Transfer Protocol}) -- protokół wymiany danymi (w pierwszej kolejności hipertekstu).
\end{description}