\chapter*{Skróty i definicje}\mbox{}\pdfbookmark[0]{Skróty}{skroty.1}
\label{sec:skroty}
\noindent
\begin{description}[labelwidth=*]
  \item [API] (ang. \ \emph{Application Programming Interface}) jest to interfejs programowania.
  \item [JSON] (ang. \ \emph{JavaScript Object Notation}) jest tekstowym formatem do wymiany danymi, którego podstawą jest Java Scrypt.
  \item [BSON] (ang. \ \emph{ Binary JavaScript Object Notation}) jest tekstowym formatem do wymiany danymi, którego podstawą jest Java Scrypt.
  \item [JWT] (ang. \ \emph{JSON Web Token}) Binarny zapis formata JSON.
  \item [SQL] (ang. \ \emph{Structured Query Language}) Jest językiem programowania strukturalnych zapytań. Najczęściej używa się do skutecznego zapisywania danych, wyshukiwania, zmiany, pobieania oraz usuwania danych z bazy. 
  \item [NoSQL] (ang. \ \emph{Not only SQL}) Szereg podejść, których celem jest tworzenie systemów zarządzania bazami dabych, któte mają dużą różnicę w porównaniu do modeli tradycyjnych relacyjnych baz dabych.
  \item [SDK] (ang. \ \emph{software development kit}) Są narzędziami programistycznymi, umozliwiające programistom tworzenie aplikacji do określianego pakietu oprogramowania. Przyznaczony do użatwiania i przespiesznia pracy z systemem.
  \item [Rest] (ang. \ \emph{Representational state transfer}) To styl achitektury oprogramowania przyznaczony do systemów rozproszonych. Zwykle używany do budowy usług internetowych.
  \item [URL] (ang. \ \emph{Uniform Resource Locator}) Jest standardem zapisu linków do obiektów w internecie.
  \item [YAML] (ang. \ \emph{Yet Another Markup Language}) Język znaczników.
  \item [UUID] (ang. \ \emph{universally unique identifier}) Standatd identyfikacji stosowany w tworzeniu oprogramowania.
  \item [HTTP] (ang. \ \emph{HyperText Transfer Protocol}) Protoków wymiany danymi (w pierwszej kolejności hipertekstu).
\end{description}