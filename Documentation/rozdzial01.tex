\chapter{Wstęp}

\section{Wprowadzenie}
Tematem pracy inżynierskiej jest aplikacja mobilna pomagająca w wyszukiwaniu miejsca ładowania samochodów elektrycznych.
Aplikacja pozwala tworzyć i wyszukiwać stacje ładownicze.

\subsubsection{Motywacja}
Dziś ludzkość coraz częściej myśli o wpływie emisji do atmosfery na ekologię Ziemi. Dotyczyło to również motoryzacji.
Według różnych źródeł, między innymi liczba samochodów elektrycznych rośnie z każdym rokiem i planuje tylko rosnąć \cite{iea1,mam1}.

W krajach WNP (Wspólnota Niepodległych Państw), i być może nie tylko w nich, problemem jest znalezienie stacji ładowania samochodów elektrycznych. Od znajomych usłyszano taką informację, że zdarzają się sytuacje,
że po przybyciu do stacji ładowania samochodu okazuje się, że stacja jest nieczynna lub znajduje się w niedostępnym miejscu, na przykład na prywatnej terytorium lub jest zbyt krótki.

Dlatego postanowiłem stworzyć aplikację, która pomogłaby w znalezieniu działających i dostępnych stacji ładowania.

Będąc w samochodzie, prawdopodobnie, że człowiek nie będzie używać ani komputera, ani laptopa. Telefon jest znacznie wygodniejszy. Dzisiaj praktycznie każdy ma telefon komórkowy.
Dlatego postanowiono stworzyć aplikację mobilną. Ponieważ smartfony z systemem Android są bardziej popularne niż Apple, dokonano wyboru aplikacji dla urządzeń z systemem Android \cite{avi1}.


\section{Cel i zakres pracy}
\subsection{Cel}
Celem pracy jest zaprojektowanie i zaimplementowanie aplikacji mobilnej z połączeniem internetowym,
pozwalającej tworzyć stację, przeglądać i wystawiać opinię o stacjach ładowania pojazdów elektrycznych.
Ta aplikacja ma na celu ułatwić wyszukiwanie miejsc do ładowania samochodów elektrycznych.

\subsection{Zakres}
Implementacja aplikacji webowej, która zawiera aplikację mobilną, części serwerowej oraz bazy danych.
Aplikacja mobilna służy wizualizacji treści oraz do komunikacji serwisu z użytkownikiem.
Część serwerowa przetwarza informację i zarządza bazą danych.

\subsection{Układ pracy}
Praca składa się z pięciu rozdziałów. W pierwszym znajduję się ogólna informacja o prace.
W drugim skupiono się na wymaganiach aplikacji, która powinna powstać oraz wykorzystanych technologiach.
Trzeci rozdział przeznaczony do opisu implementacji poszczególnych części aplikacji.
W czwartym rozdziale opisano testowanie aplikacji.
Piąty rozdział opisuję wdrożenie części serwerowej.
Na końcu pracy znajduje się podsumowanie: wnioski oraz plany na rozwój aplikacji.