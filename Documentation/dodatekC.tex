\chapter{Tytuł dodatku}
Szybki start (dodatek, w którym opisano podstawowy scenariusz użycia rozwiązania - by czytelnik mógł na podstawie tego opisu zacząć pracę z rozwiązaniem).

W szablonie, jaki opublikowałem na mojej stronie, znajduje się opis struktury pracy. Standardowo jest to:

    1. Wstęp
        1.1 Wprowadzenie
            (zwykle 1 - 2 strony wprowadzające w dziedzinę, gdzie mogą pojawić się cytowania (!), oraz zawierające wyjaśnienie odnośnie motywacji do podjęcia tematu)
        1.2 Cel i zakres pracy
            (zwykle 0,5 - 1,5 strony z opisem tego, co ma być zrobione - cel - do czego ma się przyczynić praca, zakres - co zostanie zrobione, by osiągnąć cel. Tutaj powinna być zgodność z tym, co napisano w karcie tematu pracy. Zresztą  z opisu z karty można skorzystać, lekko go przeredagowując)
        1.3 Układ pracy
            (zwykle jeden akapit na jeden rozdział z opisem zawartości danego rozdziału)
    2. Kolejny rozdział
        2.1 Sekcja
        2.1.1 Podsekcja
            Nienumerowana podpodsekcja
                Paragraf
     .....

    A. Zawartość dołączonej płyty CD/DVD
        (dodatek, w którym opisuje się zawartość nagranej płyty z projektem informatycznym oraz dokumentem pdf z tekstem pracy. Może tam się również pojawić instalator bądź obraz dockerowy - zależnie od wybranego sposobu wdrażania stworzonego rozwiązania. Generalnie chodzi o to, by recenzent sięgając po płytę miał możliwość oceny wyników pracy. Na płycie powinno być jakieś Readme)
    B. Instrukcja wdrożeniowa (dodatek, w którym opisano jak wdrożyć/zainstalować/uruchomić rozwiązanie)
    C. Szybki start (dodatek, w którym opisano podstawowy scenariusz użycia rozwiązania - by czytelnik mógł na podstawie tego opisu zacząć pracę z rozwiązaniem).

To, co pojawia się w kolejnych rozdziałach jest w dużej mierze sprawą indywidualną. Zwykle prace inżynierskie mają na początku jakieś założenia i teoretyczne podstawy, a dopiero potem opis implementacji, testów i podsumowanie. Generalnie połowa pracy powinna być o założeniach, połowa o implementacji (jeśli praca jest typowa).

W szczególności:
   - rozdział drugi może zawierać analizę wymagań: funkcjonalnych, niefunkcjonalnych, dziedzinowych plus makiety interfejsu użytkownika. Powinien w nim się pojawić opis architektury rozwiązania (wraz ze szkice jej zarysu). Poszczególne fragmentu opisu można zilustrować diagramami wymagań SysML, opisa przypadków użycia itp. Przy okazji można wspomnieć o przyjętej metodyce prowadzenia projektu
   - rozdział trzeci może zawierać przegląd wykorzystanych technologii (w rozdziale drugim można wymienić wymagania niefunkcjonalne, zaś w rozdziale trzecim podać detale), przy czym nie może to być kopia dostępnych manuali, ale syntetyczny opis tych bibliotek, narzędzi i innych rozwiązań, które wykorzystano w pracy.
   - rozdział czwarty może być o detalach implementacji. Tutaj wykorzystać można diagramy klas, przebiegów, stanów itp. Można też pokazywać fragmenty kodu (niedopuszczalne jest wstawianie całych kodów źródłowych - wystarczy pokazać interesujący fragment, wykropkowując rzeczy nieistotne)
   - rozdziął piąty może zawierać wyniki eksperymentów bądź testów
   - w ostatnim rozdziale powinno pojawić się podsumowanie, w którym jawnie będzie powiedziane w jakim stopniu udało się zrealizować cel pracy (zakładam, że uda się go zrealizować w 100%), jakie napotkano przeszkody i jak je pokonano, jakie są możliwe zastosowania stworzonego rozwiązania oraz jakie są potencjalne kierunki jego rozwoju.

Oczywiście tytuły i zawartość rozdziałów od drugiego do ostatniego nie są z góry narzucone. Wszystko zależy od charakteru pracy inżynierskiej i sposobu jej realizacji. Inaczej wygląda dokumentacja systemu, inaczej biblioteki, a jeszcze inaczej opracowanego jakiegoś szczególnego algorytmu. Zdarzają się prace inżynierskie, w których sam sposób prowadzenia projektu jest tematem wiodącym, a w związku z tym gdzie indziej stawia się akcenty i inaczej tytułuje kolejne rozdziały.

3. Reguły redakcyjne

    1. Praca dyplomowa powinna być napisana w  formie bezosobowej ("w pracy pokazano ..."). Taki styl przyjęto na uczelniach w naszym kraju, choć w krajach anglosaskich preferuje się redagowanie treści w pierwszej osobie.
    2. W tekście pracy można odwołać się do myśli autora, ale nie w pierwszej osobie, tylko poprzez wyrażenia typu: "autor wykazał, że ...".
    3. Odwołując się do rysunków i tabel należy używać zwrotów typu: "na rysunku pokazano ...", "w tabeli zamieszczono ..." (tabela i rysunek to twory nieżywotne, więc "rysunek pokazuje" jest niepoprawnym zwrotem).
    4. Praca powinna być napisana językiem formalnym, bez wyrażeń żargonowych ("sejwowanie" i "downloadowanie"), nieformalnych czy zbyt ozdobnych ("najznamienitszym przykładem tego niebywałego postępu ...")
    5. Pisząc pracę należy dbać o poprawność stylistyczną wypowiedzi
        - trzeba pamiętać, do czego stosuje się "liczba", a do czego "ilość",
        - nie "szereg funkcji" tylko "wiele funkcji",
        - redagowane zdania nie powinny być zbyt długie (lepiej podzielić zdanie wielokrotnie złożone na pojedyncze zdania),
        - itp.
    5. W pracy należy dbać o poprawność redakcyjną
        - nie zostawiać znaku spacji przed znakami interpunkcji ("powiedziano , że ..." -> "powiedziano, że ..."),
        - nie zapominać o dobrym sformatowaniu wyliczenia (należy zaczynać małymi literami lub dużymi oraz kończyć przecinkami, średnikami i kropkami - w zależności od kontekstu danego wyliczenia),
        - nie zostawiać samotnych literek na końcach linii
        - nie zostawiać pojedynczych wierszy na końcu lub początku strony (należy kontrolować "sieroty" i "wdowy")
        - czcionka na rysunkach nie powinna być większa od czcionki wiodącej,
        - na rysunkach nie wolno nadużywać kolorów oraz ozdobników (wiele narzędzi do tworzenia diagramów dostarcza grafikę z cieniowaniem, gradacją kolorów itp. - co niekoniecznie przekłada się na czytelność rysunku),
        - rysunki powinny być tworzone jako grafika wektorowa (pdf) lub rastrowa (png - z kompresją bezstratną); korzystanie z plików jpg kończy się często powstaniem tzw. halo wokół kontrastowych linii.
    6. Cytowania i wykaz literatury muszą być zredagowane poprawnie
      - w pracy dyplomowej inżynierskiej wykaz literatury zwykle ma kilkanaście pozycji (w pracach magisterskich powinno być ich więcej), z czego część może być odwołaniami do stron dotyczących wykorzystanych technologii a część powinna być "twarda" (tj. artykuły lub książki - polecam wyszukiwanie materiałów w e-czasopismach bibliteki PWr http://biblioteka.pwr.edu.pl/e-zasoby/e-czasopisma - zwykle wydawnictwa pozwalają wyeksportować metadane znalezionych pozycji w formacie bibtex lub innym). Niedopuszczalne jest korzystanie jedynie ze stron wiki i tego, co dostarcza przeglądarka google
      - niedopuszczalne są błędy literowe czy też brak metadanych dla cytowanych pozycji (zakres medatanych powinien być pełny, powinien przynajmniej posiadać: autor, tytuł, wydawca, rok wydania)
      - cytowania mają być generowane automatycznie, na podstawie przygotowanego pliku bib (parę słów o tym znajduje się w szablonie pracy).
      - należy zwrócić uwagę na zamieszczanie informacji o dacie dostępu do cytowanego źródła (jeśli to źródło jest zasobem internetowym).
      - wygenerowany wykaz literatury musi być poprawny (często przez nieuwagę w wykazie tym pojawiają się pełne imienia i nazwiska lub tylko inicjały, czasem też nazwiska są przemieszane z imionami - co zdarza się przy źle przygotowanym pliku bib)

