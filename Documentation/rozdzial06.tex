\chapter{Podsumowanie}
W tym rozdziale znajują się wnioski i plany na rozwój aplikacji.

\section{Wnioski}
% В этом проекте планировалось решить проблему с поиском зарядных станций для автомобилей. Должно было быть создано мобильное приложение, в которм пользователь может не только быстро и удобно искать походящую ему станцию, но и добавлять новые, еще не отмеченные на карте.
% Это позволило бы создать самодостаточное приложение, для которого не требовалось бы больших вложений для поддержания инфраструктуры. Приложение должно было позволить создавать коментарии и выставлять оценку зарядным станциям. Для удобства поиска должна была использоваться карта.
% На карте, кроме зарядных станций должна была отмечаться в том числе позиция пользователя.

% В приложени были реализованы все представленные выше функции, однако в процессе разработки были выявлены некоторые недостатки, о которых будет оговорено в планах на будущее. 

% Благодаря выбранныой базе данных MongoDB приложение можно легко расширять. Также все выбранные технологии оказались очень гибкими и прспособленными для этой задачи.
% Самым полезным источником информации стала официальная документация доступная на страницах \cite{android_doc,godoc,golang2,mongoDB_doc}. Книга по языку Go \cite{LearninGo}. Книга по языку Java \cite{javabook}
% % 
W tym projekcie planowano rozwiązać problem ze znalezieniem stacji ładowania pojazdów elektrycznych. Miała zostać stworzona aplikacja mobilna, w której użytkownik może nie tylko szybko i wygodnie wyszukiwać stację ładownicze, które mu odpowiadają, ale także dodawać nowe, których jeszcze nie ma w systemie.
Umożliwiłoby to stworzenie samowystarczalnej aplikacji, która nie wymagałaby dużych inwestycji w celu utrzymania infrastruktury. Aplikacja miała umożliwić tworzenie komentarzy i ocenianie stacji ładowania pojazdów elektrycznych. Dla ułatwienia wyszukiwania miała być użyta mapa.
Na mapie oprócz stacji ładowania miała być zaznaczona m.in. pozycja użytkownika.

W aplikacji zostały zaimplementowane wszystkie powyższe funkcje, ale w procesie rozwoju zidentyfikowano pewne niezbędne funkcjalności, które zostaną określone w planach \ref{sec:plany} na przyszłość.

Dzięki wybranej bazie danych MongoDB aplikacja może być łatwo rozbudowywana o nowe kolekcje i zmienione już istniejące. Wszystkie wybrane technologie okazały się bardzo elastyczne i odpowiednie przystosowane do tego zadania.
Najbardziej użytecznym źródłem informacji była oficjalna dokumentacja wybranych nażędzi \cite{android_doc,godoc,golang2,mongoDB_doc}. Też była użyteczna książka o języku Go \cite{LearninGo} oraz książka o języku Java \cite{javabook}
%
\section{Plany}
\label{sec:plany}
% % 
% Безопасность: 
% \begin{itemize}
%     \item Создание панели администратора для модерации комментариев;
%     \item Закрытие доступа любого авторезированого пользователя к данным, которые не разрешены через мобильное приложение.
% \end{itemize}

% Удобство пользования: 
% \begin{itemize}
%     \item Добавление возможности удаления комментариев через мобильное приложение;
%     \item Расширение возможностей работы с картой с помощью сервисов google cloud. Например прокладывание маршрута;
% \end{itemize}
Bezpieczeństwo:
\begin{itemize}
    \item Tworzenie panelu admina do moderacji komentarzy;
    \item Zamknięcie dostępu dowolnego użytkownika do danych, które nie są dozwolone za pośrednictwem aplikacji mobilnej, to znaczy zamknięcie API;
    \item Uleprzenie niektórych endpointów, na przykład sposobu wyszukiwania wedłóg parametrów w endpoincie \texttt{/api/v1/stations/read?skip=""\&limit=""\&lat=""\&lng=""\&dist=""\&descr=""\&nam=""}.
\end{itemize}

Wygoda użytkowania aplikacją:
\begin{itemize}
    \item Dodawanie możliwości usuwania komentarzy za pośrednictwem aplikacji mobilnej;
    \item Rozszerzenie możliwości pracy z mapą za pomocą usług google cloud. Na przykład wyznaczanie trasy;
\end{itemize}